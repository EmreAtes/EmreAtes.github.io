%%%%%%%%%%%%%%%%%%%%%%%%%%%%%%%%%%%%%%%%%
% Medium Length Professional CV
% LaTeX Template
% Version 2.0 (8/5/13)
%
% This template has been downloaded from:
% http://www.LaTeXTemplates.com
%
% Original author:
% Trey Hunner (http://www.treyhunner.com/)
%
% Important note:
% This template requires the resume.cls file to be in the same directory as the
% .tex file. The resume.cls file provides the resume style used for structuring the
% document.
%
%%%%%%%%%%%%%%%%%%%%%%%%%%%%%%%%%%%%%%%%%

%----------------------------------------------------------------------------------------
%	PACKAGES AND OTHER DOCUMENT CONFIGURATIONS
%----------------------------------------------------------------------------------------

\documentclass{resume}

\usepackage[left=0.75in,top=0.6in,right=0.75in,bottom=1in]{geometry} % Document margins
\usepackage{hyperref}
\usepackage{fancyhdr}
\usepackage{lastpage}
\hypersetup{backref,pdfpagemode=Full,colorlinks=false,backref}
\newcommand{\tab}[1]{\hspace{.2667\textwidth}\rlap{#1}}
\newcommand{\itab}[1]{\hspace{0em}\rlap{#1}}
\def\Cplusplus{{\rm C\raise.25ex\hbox{\small ++}}}
\def\bfCplusplus{{\rm{\bf C\raise.25ex\hbox{\small ++}}}}


\fancypagestyle{empty}{
\fancyhf{}
\renewcommand{\headrule}{}
\fancyfoot[r]{Emre Ates \\ Page \thepage \hspace{1pt} of \pageref{LastPage}}}
\pagestyle{empty}

\name{Emre Ate\c{s}}
\address{(857) 540-8435 \\ \href{mailto:ates.emre@gmail.com}{ates.emre@gmail.com} \\
  \href{https://emreates.github.io}{https://emreates.github.io}}

\begin{document}
% \thispagestyle{plain}

%----------------------------------------------------------------------------------------
%	EDUCATION SECTION
%----------------------------------------------------------------------------------------

\begin{rSection}{Education}

\begin{rSubsection}{Boston University}{Summer 2020}{PhD in
    Computer Engineering (Advisor: Prof. Ay\c{s}e K. Co\c{s}kun)}{GPA: 3.93 / 4.0}
\item \textbf{Thesis title:} Towards automated analytics on large-scale
  computing systems
\item \textbf{Coursework:} Data Structures and Algorithms, Computer
  Architecture, Data Mining, \\
  Operating Systems, Cybersecurity, Computer Systems, Digital Design, Embedded
  Systems
\end{rSubsection}

\begin{rSubsectionNoList}{Middle East Technical University (METU),
    Turkey}{Spring 2015}{BSc in Electrical and Electronics Engineering}{}{Minor
    in History of Philosophy}
\end{rSubsectionNoList}

\end{rSection}
%----------------------------------------------------------------------------------------
%	TECHNICAL STRENGTHS SECTION
%----------------------------------------------------------------------------------------

\begin{rSection}{Technical Skills}

\begin{tabular}{ @{} >{\bfseries}l @{\hspace{6ex}} l }
Languages & {\em (proficient:)} C, \Cplusplus, Python, Rust, Bash, {\em (familiar:)} SQL, R, Java, Perl \\
Software \& Tools & git, gdb, OpenStack, scikit-learn, Vowpal Wabbit, Autotools, TensorFlow \\
\end{tabular}

\end{rSection}

%----------------------------------------------------------------------------------------
%	WORK EXPERIENCE SECTION
%----------------------------------------------------------------------------------------

\begin{rSection}{Work Experience}

\begin{rWorksection}{Google, Boston}{August 2020 -- present}{Software Engineer}
\item Improve video quality and reduce bitrates for YouTube using hardware
    accelerators: \href{http://goo.gle/vcu}{\nolinkurl{goo.gle/vcu}}.
\end{rWorksection}

\begin{rWorksection}{Google, NYC}{Spring 2019}{Software Engineering
    Internship}
\item Implemented data collection and heuristics in {\bf \bfCplusplus{}, Go} within the memory
  allocator, TCMalloc.
\item Built a simulator pipeline using {\bf SQL, \bfCplusplus{}, Flume} to
  compare heuristics.
\item Improved the performance of a major Google service in data center-scale tests.
\end{rWorksection}

\begin{rWorksection}{Lawrence Livermore National Laboratory}{Summer
    2017}{Research Internship}
\item Measured performance effects of power/network on supercomputers using {\bf
    Bash} and {\bf Python}.
\item Improved compatibility of power measurement {\bf kernel module} for the
  latest version of Linux.
\end{rWorksection}

% \begin{rWorksection}{Boston University}{Spring 2016, Fall 2016}
%     {Head Teaching Assistant}
% \item Held weekly discussion sessions, wrote homework solutions for Applied
%   Algorithms and Data Structures.
% \end{rWorksection}

\begin{rWorksection}{Sandia National Laboratories}{Summer 2016}{Research
    Internship}
\item Studied network contention on application performance for HPC systems
  using {\bf MPI}.
\end{rWorksection}

\end{rSection}

\begin{rSection}{Select Projects}

\begin{rWorksection}{HPC Performance Analytics}{2015 -- 2020}{Boston
    University \& Sandia National Labs}
\item Developed an HPC performance interference generation suite in {\bf C}.
\item Built a supervised learning framework in {\bf Python} using {\bf MongoDB,
    scikit-learn, TensorFlow} that collects numeric time series data from
  supercomputers, and detects performance anomalies, running applications, or
  cryptocurrency mining.
\end{rWorksection}

\begin{rWorksection}{Distributed Tracing on the Cloud}{2017 -- 2020}{Boston
    University \& RedHat}
\item Extended existing distributed tracing for {\bf OpenStack} using {\bf
    Python, Redis}.
\item Built a graph processing pipeline in {\bf Rust} to explore
  instrumentation options in response to ongoing performance problems.
\end{rWorksection}

\end{rSection}

\pagebreak

%----------------------------------------------------------------------------------------
%	PUBLICATIONS
%----------------------------------------------------------------------------------------

\begin{rSection}{Publications} \itemsep -3pt
\item \textbf{E. Ates}, B. Aksar, V.J. Leung, A.K. Coskun ``Counterfactual
Explanations for Multivariate Time Series.'' to appear in \textit{International
  Conference on Applied Artificial Intelligence} (ICAPAI), 2021.

\item A. Byrne, \textbf{E. Ates}, A. Turk, V. Pchelin, S. Duri, S. Nadgowda, C.
Isci, A.K. Coskun, ``Praxi: Cloud software discovery that learns from
practice,'' to appear in \textit{IEEE Trans. on Cloud Computing} (TCC).

\item \textbf{E. Ates}, L. Sturmann, M. Toslali, O. Krieger, R. Megginson, A.K.
Coskun, R.R. Sambasivan, ``An automated, cross-layer instrumentation framework
for diagnosing performance problems in distributed applications,'' in
\textit{Symposium on Cloud Computing} (SoCC), Santa Cruz, 2019.

\item \textbf{E. Ates}, Y. Zhang, B. Aksar, J. Brandt, V.J. Leung, M.
Egele, A.K. Coskun, ``HPAS: An HPC performance anomaly suite for reproducing
performance variations,'' in \textit{Intl. Conf. on Parallel Processing}
(ICPP), Kyoto, 2019.

\item O. Tuncer, \textbf{E. Ates}, Y. Zhang, A. Turk, J. Brandt, V.J. Leung, M.
Egele, A.K. Coskun, ``Online diagnosis of performance variation in HPC systems
using machine learning,'' in \textit{IEEE Trans. on Parallel and
  Distributed Systems}, vol. 30, no. 4, pp. 883-896, 2019.

\item Q. Xiong, \textbf{E. Ates}, M.C. Herbordt, A.K. Coskun, ``Tangram: Colocating
HPC applications with oversubscription,'' in \textit{IEEE High Performance
  Extreme Computing Conf.}, Boston, 2018.

\item \textbf{E. Ates}, O. Tuncer, A. Turk, J. Brandt, V.J. Leung, M. Egele, A.K.
Coskun, ``Taxonomist: Application detection through rich monitoring data,'' in
\textit{European Conf. on Parallel and Distributed Systems} (Euro-Par),
Torino, 2018.

\item T. Patki, \textbf{E. Ates}, A.K. Coskun, J.J. Thiagarajan, ``Understanding
simultaneous impact of network QoS and power on HPC application performance,''
in \textit{Computational Reproducibility at Exascale} (CRE), Dallas, 2018.

\item O. Tuncer, \textbf{E. Ates}, Y. Zhang, A. Turk, J. Brandt, V.J. Leung, M.
Egele, A.K. Coskun, ``Diagnosing performance variations in HPC applications
using machine learning,'' in \textit{Intl. Supercomputing Conf.} (ISC-HPC),
Frankfurt, 2017.

\end{rSection}

%----------------------------------------------------------------------------------------
\begin{rSection}{Awards and Fellowships} \itemsep -3pt
  \item Best Artifact Award at EuroPar'18
  \item Gauss Center for Supercomputing Award at ISC-HPC'17
  \item A. Richard Newton Young Fellowship at DAC'16
  \item Distinguished ECE Fellowship from Boston University
  \item Analog Electronics Laboratory Best Project Award at METU
\end{rSection}

\begin{rSection}{Activities}
  \begin{rWorksectionNoList}{Student Volunteer}{November 2019}{Symposium on Cloud
      Computing (SoCC)}
  \end{rWorksectionNoList}

  \begin{rWorksectionNoList}{Student Volunteer}{October 2017}{International
      Conference for High Performance Computing, Networking, Storage and
      Analysis (SC)}
  \end{rWorksectionNoList}

  \begin{rWorksectionNoList}{Pianist}{2015 -- 2018}{Boston University Big Band}
  \end{rWorksectionNoList}

    \begin{rWorksection}{Musical Director \& Pianist}{2012 -- 2013}{
      METU Musical Society}
  \item Led a team of 12 instrumentalists, and trained 14 actors to stage
    multiple Broadway musicals.
  \item Collaborated with professionals from all branches of show business, and
    a technical crew of 30.
  \end{rWorksection}
\end{rSection}

\end{document}

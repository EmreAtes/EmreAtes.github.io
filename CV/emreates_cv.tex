%%%%%%%%%%%%%%%%%%%%%%%%%%%%%%%%%%%%%%%%%
% Medium Length Professional CV
% LaTeX Template
% Version 2.0 (8/5/13)
%
% This template has been downloaded from:
% http://www.LaTeXTemplates.com
%
% Original author:
% Trey Hunner (http://www.treyhunner.com/)
%
% Important note:
% This template requires the resume.cls file to be in the same directory as the
% .tex file. The resume.cls file provides the resume style used for structuring the
% document.
%
%%%%%%%%%%%%%%%%%%%%%%%%%%%%%%%%%%%%%%%%%

%----------------------------------------------------------------------------------------
%	PACKAGES AND OTHER DOCUMENT CONFIGURATIONS
%----------------------------------------------------------------------------------------

\documentclass{resume}

\usepackage[left=0.75in,top=0.6in,right=0.75in,bottom=0.6in]{geometry} % Document margins
\usepackage{hyperref}
\hypersetup{backref,pdfpagemode=Full,colorlinks=false,backref}
\newcommand{\tab}[1]{\hspace{.2667\textwidth}\rlap{#1}}
\newcommand{\itab}[1]{\hspace{0em}\rlap{#1}}
\def\Cplusplus{{\rm C\raise.25ex\hbox{\small ++}}}
\def\bfCplusplus{{\rm{\bf C\raise.25ex\hbox{\small ++}}}}
\name{Emre Ate\c{s}}
\address{Boston, MA}
\address{(+1) 857 540 8435 \\ \href{mailto:ates@bu.edu}{ates@bu.edu} \\ \url{https://emreates.github.io}}

\begin{document}

%----------------------------------------------------------------------------------------
%	EDUCATION SECTION
%----------------------------------------------------------------------------------------

\begin{rSection}{Education}

\begin{rSubsection}{Boston University}{2015 - Summer 2020 (Expected)}{PhD in
    Computer Engineering (Advisor: Prof. Ay\c{s}e K. Co\c{s}kun)}{GPA: 3.93}
\item \textbf{Coursework:} Data Structures and Algorithms, Computer
  Architecture, Data Mining, Operating Systems, Cybersecurity, Computer Systems,
  Digital Design, Embedded Systems
\end{rSubsection}

\begin{rSubsection}{Middle East Technical University (METU), Turkey}{2010 -
    2015}{BSc in Electrical and Electronics Engineering, Minor in History of
    Philosophy}{GPA: 3.23, top 10\%}
\end{rSubsection}

\end{rSection}
%----------------------------------------------------------------------------------------
%	TECHNICAL STRENGTHS SECTION
%----------------------------------------------------------------------------------------

\begin{rSection}{Technical Strengths}

\begin{tabular}{ @{} >{\bfseries}l @{\hspace{6ex}} l }
Languages & {\em (proficient:)} C, \Cplusplus, Python, Rust, Bash, {\em (familiar:)} SQL, R, Java, Perl \\
Software \& Tools & git, gdb, OpenStack, scikit-learn, Vowpal Wabbit, Autotools, tensorflow \\
\end{tabular}

\end{rSection}

%----------------------------------------------------------------------------------------
%	WORK EXPERIENCE SECTION
%----------------------------------------------------------------------------------------

\begin{rSection}{Experience}

\begin{rWorksection}{Boston University, PeacLab}{Fall 2015 -- present}{Research Assistant}
\item Researched on improving large-scale computing systems using advanced
  analytics approaches.
  % Topics:
  % \begin{itemize}
  % \item Data center monitoring and analytics using machine learning,
  % \item End-to-end tracing of distributed applications,
  % \item Machine learning on filesystem changes.
  % \end{itemize}
\end{rWorksection}

\begin{rWorksection}{Google, NYC}{Spring 2019}{Software Engineering
    Internship}
\item As part of the Google Wide Profiling team,
  \begin{list}{$\cdot$}{\leftmargin=2em} % \cdot used for bullets, no indentation
    \itemsep -0.5em \vspace{-0.5em} % Compress items in list together for aesthetics
  \item Implemented multiple heuristics in {\bf \bfCplusplus{}} within the memory
    allocator, TCMalloc.
  \item Implemented collection of metrics from TCMalloc users using {\bf
      \bfCplusplus{}, Go}.
  \item Built a simulator pipeline using {\bf SQL, \bfCplusplus{}, Flume} to
    compare various heuristics.
  \end{list}
\end{rWorksection}

\begin{rWorksection}{Lawrence Livermore National Laboratory}{Summer
    2017}{Research Internship}
\item Ran comprehensive benchmarks in supercomputers on the effects of
  power/network on performance using {\bf Bash, Python}.
\end{rWorksection}

\begin{rWorksection}{Sandia National Laboratories}{Summer 2016}{Research
    Internship}
\item Studied network contention on application performance for HPC systems
  using {\bf MPI}.
\end{rWorksection}

\end{rSection}


%	EXAMPLE SECTION
%----------------------------------------------------------------------------------------

\begin{rSection}{Selected Academic Projects} \itemsep -2pt
\item {\bf HPC Performance Anomaly Diagnosis:}
  \begin{list}{$\cdot$}{\leftmargin=2em} % \cdot used for bullets, no indentation
    \itemsep -0.5em \vspace{-0.5em} % Compress items in list together for aesthetics
  \item Developed an HPC performance interference generation suite in {\bf C},
  \item Collected numeric time series data from supercomputers,
  \item Built a supervised learning pipeline in {\bf Python} to detect
    performance anomalies using the time series data using {\bf MongoDB,
      scikit-learn, Tensorflow}.
  \end{list}
  \item {\bf Distributed Tracing on the Cloud:}
  \begin{list}{$\cdot$}{\leftmargin=2em} % \cdot used for bullets, no indentation
    \itemsep -0.5em \vspace{-0.5em} % Compress items in list together for aesthetics
  \item Extended existing distributed tracing for {\bf OpenStack} using {\bf
      Python, Redis},
  \item Built a graph processing pipeline in {\bf Rust} to explore
    instrumentation options in response to ongoing performance problems.
  \end{list}
\end{rSection}

%----------------------------------------------------------------------------------------
%	PUBLICATIONS
%----------------------------------------------------------------------------------------

\begin{rSection}{Publications} \itemsep -3pt
\item \textbf{E. Ates}, L. Sturmann, M. Toslali, O. Krieger, R. Megginson, A.K.
Coskun, R.R. Sambasivan, ``An automated, cross-layer instrumentation framework
for diagnosing performance problems in distributed applications,'' in
\textit{Symposium on Cloud Computing} (SoCC), Santa Cruz, 2019.

\item \textbf{E. Ates}, Y. Zhang, B. Aksar, J. Brandt, V.J. Leung, M.
Egele, A.K. Coskun, ``HPAS: An HPC Performance Anomaly Suite for Reproducing
Performance Variations,'' in \textit{Intl. Conf. on Parallel Processing}
(ICPP), Kyoto, 2019.

\item O. Tuncer, \textbf{E. Ates}, Y. Zhang, A. Turk, J. Brandt, V.J. Leung, M.
Egele, A.K. Coskun, ``Online Diagnosis of Performance Variation in HPC Systems
Using Machine Learning,'' in \textit{IEEE Trans. on Parallel and
  Distributed Systems}, vol. 30, no. 4, pp. 883-896, 2019.

\item Q. Xiong, \textbf{E. Ates}, M.C. Herbordt, A.K. Coskun, ``Tangram: Colocating
HPC Applications with Oversubscription,'' in \textit{IEEE High Performance
  Extreme Computing Conf.}, Boston, 2018.

\item \textbf{E. Ates}, O. Tuncer, A. Turk, J. Brandt, V.J. Leung, M. Egele, A.K.
Coskun, ``Taxonomist: Application Detection through Rich Monitoring Data,'' in
\textit{European Conf. on Parallel and Distributed Systems} (Euro-Par),
Torino, 2018.

\item T. Patki, \textbf{E. Ates}, A.K. Coskun, J.J. Thiagarajan, ``Understanding
Simultaneous Impact of Network QoS and Power on HPC Application Performance,''
in \textit{Computational Reproducibility at Exascale} (CRE), Dallas, 2018.

\item O. Tuncer, \textbf{E. Ates}, Y. Zhang, A. Turk, J. Brandt, V.J. Leung, M.
Egele, A.K. Coskun, ``Diagnosing Performance Variations in HPC Applications
using Machine Learning,'' in \textit{Intl. Supercomputing Conf.} (ISC-HPC),
Frankfurt, 2017.

\end{rSection}

%----------------------------------------------------------------------------------------
\begin{rSection}{Other} \itemsep -3pt
\item {\bf Awards and Fellowships:} Best Artifact Award at EuroPar'18, Gauss
  Center for Supercomputing Award at ISC-HPC'17, A. Richard Newton Young
  Fellowship at DAC'16, Distinguished ECE Fellowship from Boston University,
  Analog Electronics Laboratory Best Project Award at METU.
\item {\bf Student Volunteer:} At SC'17 and SoCC'19.
\item {\bf Teaching:} {\bf Head Teaching Assistant} for Applied Algorithms and Data
  Stuctures at Boston University (Spring 2016, Fall 2016). Held weekly
  discussion sessions, graded exams/assignments, coordinated the graders. {\bf
    Instructor} for BU Summer Challenge (2018). Taught introductory electrical
  engineering to high school students.
\item {\bf Open Source Projects:} \url{https://github.com/peaclab/hpas},
  \url{https://doi.org/10.6084/m9.figshare.6384248},
  \url{https://github.com/uuid-rs/uuid-gdb}
\item {\bf Pianist} (2010 - 2015) and {\bf musical director} (2012 - 2013) of
  METU Musical Society \\
  Led a team of 12 instrumentalists, and trained 14 actors to stage multiple
  Broadway musicals in METU, collaborating with professionals from all branches
  of show business, and a technical crew of 30

\end{rSection}

\end{document}
